\documentclass[11pt]{article}

    \usepackage[breakable]{tcolorbox}
    \usepackage{parskip} % Stop auto-indenting (to mimic markdown behaviour)
    
    \usepackage{iftex}
    \ifPDFTeX
    	\usepackage[T1]{fontenc}
    	\usepackage{mathpazo}
    \else
    	\usepackage{fontspec}
    \fi

    % Basic figure setup, for now with no caption control since it's done
    % automatically by Pandoc (which extracts ![](path) syntax from Markdown).
    \usepackage{graphicx}
    % Maintain compatibility with old templates. Remove in nbconvert 6.0
    \let\Oldincludegraphics\includegraphics
    % Ensure that by default, figures have no caption (until we provide a
    % proper Figure object with a Caption API and a way to capture that
    % in the conversion process - todo).
    \usepackage{caption}
    \DeclareCaptionFormat{nocaption}{}
    \captionsetup{format=nocaption,aboveskip=0pt,belowskip=0pt}

    \usepackage[Export]{adjustbox} % Used to constrain images to a maximum size
    \adjustboxset{max size={0.9\linewidth}{0.9\paperheight}}
    \usepackage{float}
    \floatplacement{figure}{H} % forces figures to be placed at the correct location
    \usepackage{xcolor} % Allow colors to be defined
    \usepackage{enumerate} % Needed for markdown enumerations to work
    \usepackage{geometry} % Used to adjust the document margins
    \usepackage{amsmath} % Equations
    \usepackage{amssymb} % Equations
    \usepackage{textcomp} % defines textquotesingle
    % Hack from http://tex.stackexchange.com/a/47451/13684:
    \AtBeginDocument{%
        \def\PYZsq{\textquotesingle}% Upright quotes in Pygmentized code
    }
    \usepackage{upquote} % Upright quotes for verbatim code
    \usepackage{eurosym} % defines \euro
    \usepackage[mathletters]{ucs} % Extended unicode (utf-8) support
    \usepackage{fancyvrb} % verbatim replacement that allows latex
    \usepackage{grffile} % extends the file name processing of package graphics 
                         % to support a larger range
    \makeatletter % fix for grffile with XeLaTeX
    \def\Gread@@xetex#1{%
      \IfFileExists{"\Gin@base".bb}%
      {\Gread@eps{\Gin@base.bb}}%
      {\Gread@@xetex@aux#1}%
    }
    \makeatother

    % The hyperref package gives us a pdf with properly built
    % internal navigation ('pdf bookmarks' for the table of contents,
    % internal cross-reference links, web links for URLs, etc.)
    \usepackage{hyperref}
    % The default LaTeX title has an obnoxious amount of whitespace. By default,
    % titling removes some of it. It also provides customization options.
    \usepackage{titling}
    \usepackage{longtable} % longtable support required by pandoc >1.10
    \usepackage{booktabs}  % table support for pandoc > 1.12.2
    \usepackage[inline]{enumitem} % IRkernel/repr support (it uses the enumerate* environment)
    \usepackage[normalem]{ulem} % ulem is needed to support strikethroughs (\sout)
                                % normalem makes italics be italics, not underlines
    \usepackage{mathrsfs}
    

    
    % Colors for the hyperref package
    \definecolor{urlcolor}{rgb}{0,.145,.698}
    \definecolor{linkcolor}{rgb}{.71,0.21,0.01}
    \definecolor{citecolor}{rgb}{.12,.54,.11}

    % ANSI colors
    \definecolor{ansi-black}{HTML}{3E424D}
    \definecolor{ansi-black-intense}{HTML}{282C36}
    \definecolor{ansi-red}{HTML}{E75C58}
    \definecolor{ansi-red-intense}{HTML}{B22B31}
    \definecolor{ansi-green}{HTML}{00A250}
    \definecolor{ansi-green-intense}{HTML}{007427}
    \definecolor{ansi-yellow}{HTML}{DDB62B}
    \definecolor{ansi-yellow-intense}{HTML}{B27D12}
    \definecolor{ansi-blue}{HTML}{208FFB}
    \definecolor{ansi-blue-intense}{HTML}{0065CA}
    \definecolor{ansi-magenta}{HTML}{D160C4}
    \definecolor{ansi-magenta-intense}{HTML}{A03196}
    \definecolor{ansi-cyan}{HTML}{60C6C8}
    \definecolor{ansi-cyan-intense}{HTML}{258F8F}
    \definecolor{ansi-white}{HTML}{C5C1B4}
    \definecolor{ansi-white-intense}{HTML}{A1A6B2}
    \definecolor{ansi-default-inverse-fg}{HTML}{FFFFFF}
    \definecolor{ansi-default-inverse-bg}{HTML}{000000}

    % commands and environments needed by pandoc snippets
    % extracted from the output of `pandoc -s`
    \providecommand{\tightlist}{%
      \setlength{\itemsep}{0pt}\setlength{\parskip}{0pt}}
    \DefineVerbatimEnvironment{Highlighting}{Verbatim}{commandchars=\\\{\}}
    % Add ',fontsize=\small' for more characters per line
    \newenvironment{Shaded}{}{}
    \newcommand{\KeywordTok}[1]{\textcolor[rgb]{0.00,0.44,0.13}{\textbf{{#1}}}}
    \newcommand{\DataTypeTok}[1]{\textcolor[rgb]{0.56,0.13,0.00}{{#1}}}
    \newcommand{\DecValTok}[1]{\textcolor[rgb]{0.25,0.63,0.44}{{#1}}}
    \newcommand{\BaseNTok}[1]{\textcolor[rgb]{0.25,0.63,0.44}{{#1}}}
    \newcommand{\FloatTok}[1]{\textcolor[rgb]{0.25,0.63,0.44}{{#1}}}
    \newcommand{\CharTok}[1]{\textcolor[rgb]{0.25,0.44,0.63}{{#1}}}
    \newcommand{\StringTok}[1]{\textcolor[rgb]{0.25,0.44,0.63}{{#1}}}
    \newcommand{\CommentTok}[1]{\textcolor[rgb]{0.38,0.63,0.69}{\textit{{#1}}}}
    \newcommand{\OtherTok}[1]{\textcolor[rgb]{0.00,0.44,0.13}{{#1}}}
    \newcommand{\AlertTok}[1]{\textcolor[rgb]{1.00,0.00,0.00}{\textbf{{#1}}}}
    \newcommand{\FunctionTok}[1]{\textcolor[rgb]{0.02,0.16,0.49}{{#1}}}
    \newcommand{\RegionMarkerTok}[1]{{#1}}
    \newcommand{\ErrorTok}[1]{\textcolor[rgb]{1.00,0.00,0.00}{\textbf{{#1}}}}
    \newcommand{\NormalTok}[1]{{#1}}
    
    % Additional commands for more recent versions of Pandoc
    \newcommand{\ConstantTok}[1]{\textcolor[rgb]{0.53,0.00,0.00}{{#1}}}
    \newcommand{\SpecialCharTok}[1]{\textcolor[rgb]{0.25,0.44,0.63}{{#1}}}
    \newcommand{\VerbatimStringTok}[1]{\textcolor[rgb]{0.25,0.44,0.63}{{#1}}}
    \newcommand{\SpecialStringTok}[1]{\textcolor[rgb]{0.73,0.40,0.53}{{#1}}}
    \newcommand{\ImportTok}[1]{{#1}}
    \newcommand{\DocumentationTok}[1]{\textcolor[rgb]{0.73,0.13,0.13}{\textit{{#1}}}}
    \newcommand{\AnnotationTok}[1]{\textcolor[rgb]{0.38,0.63,0.69}{\textbf{\textit{{#1}}}}}
    \newcommand{\CommentVarTok}[1]{\textcolor[rgb]{0.38,0.63,0.69}{\textbf{\textit{{#1}}}}}
    \newcommand{\VariableTok}[1]{\textcolor[rgb]{0.10,0.09,0.49}{{#1}}}
    \newcommand{\ControlFlowTok}[1]{\textcolor[rgb]{0.00,0.44,0.13}{\textbf{{#1}}}}
    \newcommand{\OperatorTok}[1]{\textcolor[rgb]{0.40,0.40,0.40}{{#1}}}
    \newcommand{\BuiltInTok}[1]{{#1}}
    \newcommand{\ExtensionTok}[1]{{#1}}
    \newcommand{\PreprocessorTok}[1]{\textcolor[rgb]{0.74,0.48,0.00}{{#1}}}
    \newcommand{\AttributeTok}[1]{\textcolor[rgb]{0.49,0.56,0.16}{{#1}}}
    \newcommand{\InformationTok}[1]{\textcolor[rgb]{0.38,0.63,0.69}{\textbf{\textit{{#1}}}}}
    \newcommand{\WarningTok}[1]{\textcolor[rgb]{0.38,0.63,0.69}{\textbf{\textit{{#1}}}}}
    
    
    % Define a nice break command that doesn't care if a line doesn't already
    % exist.
    \def\br{\hspace*{\fill} \\* }
    % Math Jax compatibility definitions
    \def\gt{>}
    \def\lt{<}
    \let\Oldtex\TeX
    \let\Oldlatex\LaTeX
    \renewcommand{\TeX}{\textrm{\Oldtex}}
    \renewcommand{\LaTeX}{\textrm{\Oldlatex}}
    % Document parameters
    % Document title
    \title{Red Generativa Adversaria}
    
    
    
    
    
% Pygments definitions
\makeatletter
\def\PY@reset{\let\PY@it=\relax \let\PY@bf=\relax%
    \let\PY@ul=\relax \let\PY@tc=\relax%
    \let\PY@bc=\relax \let\PY@ff=\relax}
\def\PY@tok#1{\csname PY@tok@#1\endcsname}
\def\PY@toks#1+{\ifx\relax#1\empty\else%
    \PY@tok{#1}\expandafter\PY@toks\fi}
\def\PY@do#1{\PY@bc{\PY@tc{\PY@ul{%
    \PY@it{\PY@bf{\PY@ff{#1}}}}}}}
\def\PY#1#2{\PY@reset\PY@toks#1+\relax+\PY@do{#2}}

\expandafter\def\csname PY@tok@w\endcsname{\def\PY@tc##1{\textcolor[rgb]{0.73,0.73,0.73}{##1}}}
\expandafter\def\csname PY@tok@c\endcsname{\let\PY@it=\textit\def\PY@tc##1{\textcolor[rgb]{0.25,0.50,0.50}{##1}}}
\expandafter\def\csname PY@tok@cp\endcsname{\def\PY@tc##1{\textcolor[rgb]{0.74,0.48,0.00}{##1}}}
\expandafter\def\csname PY@tok@k\endcsname{\let\PY@bf=\textbf\def\PY@tc##1{\textcolor[rgb]{0.00,0.50,0.00}{##1}}}
\expandafter\def\csname PY@tok@kp\endcsname{\def\PY@tc##1{\textcolor[rgb]{0.00,0.50,0.00}{##1}}}
\expandafter\def\csname PY@tok@kt\endcsname{\def\PY@tc##1{\textcolor[rgb]{0.69,0.00,0.25}{##1}}}
\expandafter\def\csname PY@tok@o\endcsname{\def\PY@tc##1{\textcolor[rgb]{0.40,0.40,0.40}{##1}}}
\expandafter\def\csname PY@tok@ow\endcsname{\let\PY@bf=\textbf\def\PY@tc##1{\textcolor[rgb]{0.67,0.13,1.00}{##1}}}
\expandafter\def\csname PY@tok@nb\endcsname{\def\PY@tc##1{\textcolor[rgb]{0.00,0.50,0.00}{##1}}}
\expandafter\def\csname PY@tok@nf\endcsname{\def\PY@tc##1{\textcolor[rgb]{0.00,0.00,1.00}{##1}}}
\expandafter\def\csname PY@tok@nc\endcsname{\let\PY@bf=\textbf\def\PY@tc##1{\textcolor[rgb]{0.00,0.00,1.00}{##1}}}
\expandafter\def\csname PY@tok@nn\endcsname{\let\PY@bf=\textbf\def\PY@tc##1{\textcolor[rgb]{0.00,0.00,1.00}{##1}}}
\expandafter\def\csname PY@tok@ne\endcsname{\let\PY@bf=\textbf\def\PY@tc##1{\textcolor[rgb]{0.82,0.25,0.23}{##1}}}
\expandafter\def\csname PY@tok@nv\endcsname{\def\PY@tc##1{\textcolor[rgb]{0.10,0.09,0.49}{##1}}}
\expandafter\def\csname PY@tok@no\endcsname{\def\PY@tc##1{\textcolor[rgb]{0.53,0.00,0.00}{##1}}}
\expandafter\def\csname PY@tok@nl\endcsname{\def\PY@tc##1{\textcolor[rgb]{0.63,0.63,0.00}{##1}}}
\expandafter\def\csname PY@tok@ni\endcsname{\let\PY@bf=\textbf\def\PY@tc##1{\textcolor[rgb]{0.60,0.60,0.60}{##1}}}
\expandafter\def\csname PY@tok@na\endcsname{\def\PY@tc##1{\textcolor[rgb]{0.49,0.56,0.16}{##1}}}
\expandafter\def\csname PY@tok@nt\endcsname{\let\PY@bf=\textbf\def\PY@tc##1{\textcolor[rgb]{0.00,0.50,0.00}{##1}}}
\expandafter\def\csname PY@tok@nd\endcsname{\def\PY@tc##1{\textcolor[rgb]{0.67,0.13,1.00}{##1}}}
\expandafter\def\csname PY@tok@s\endcsname{\def\PY@tc##1{\textcolor[rgb]{0.73,0.13,0.13}{##1}}}
\expandafter\def\csname PY@tok@sd\endcsname{\let\PY@it=\textit\def\PY@tc##1{\textcolor[rgb]{0.73,0.13,0.13}{##1}}}
\expandafter\def\csname PY@tok@si\endcsname{\let\PY@bf=\textbf\def\PY@tc##1{\textcolor[rgb]{0.73,0.40,0.53}{##1}}}
\expandafter\def\csname PY@tok@se\endcsname{\let\PY@bf=\textbf\def\PY@tc##1{\textcolor[rgb]{0.73,0.40,0.13}{##1}}}
\expandafter\def\csname PY@tok@sr\endcsname{\def\PY@tc##1{\textcolor[rgb]{0.73,0.40,0.53}{##1}}}
\expandafter\def\csname PY@tok@ss\endcsname{\def\PY@tc##1{\textcolor[rgb]{0.10,0.09,0.49}{##1}}}
\expandafter\def\csname PY@tok@sx\endcsname{\def\PY@tc##1{\textcolor[rgb]{0.00,0.50,0.00}{##1}}}
\expandafter\def\csname PY@tok@m\endcsname{\def\PY@tc##1{\textcolor[rgb]{0.40,0.40,0.40}{##1}}}
\expandafter\def\csname PY@tok@gh\endcsname{\let\PY@bf=\textbf\def\PY@tc##1{\textcolor[rgb]{0.00,0.00,0.50}{##1}}}
\expandafter\def\csname PY@tok@gu\endcsname{\let\PY@bf=\textbf\def\PY@tc##1{\textcolor[rgb]{0.50,0.00,0.50}{##1}}}
\expandafter\def\csname PY@tok@gd\endcsname{\def\PY@tc##1{\textcolor[rgb]{0.63,0.00,0.00}{##1}}}
\expandafter\def\csname PY@tok@gi\endcsname{\def\PY@tc##1{\textcolor[rgb]{0.00,0.63,0.00}{##1}}}
\expandafter\def\csname PY@tok@gr\endcsname{\def\PY@tc##1{\textcolor[rgb]{1.00,0.00,0.00}{##1}}}
\expandafter\def\csname PY@tok@ge\endcsname{\let\PY@it=\textit}
\expandafter\def\csname PY@tok@gs\endcsname{\let\PY@bf=\textbf}
\expandafter\def\csname PY@tok@gp\endcsname{\let\PY@bf=\textbf\def\PY@tc##1{\textcolor[rgb]{0.00,0.00,0.50}{##1}}}
\expandafter\def\csname PY@tok@go\endcsname{\def\PY@tc##1{\textcolor[rgb]{0.53,0.53,0.53}{##1}}}
\expandafter\def\csname PY@tok@gt\endcsname{\def\PY@tc##1{\textcolor[rgb]{0.00,0.27,0.87}{##1}}}
\expandafter\def\csname PY@tok@err\endcsname{\def\PY@bc##1{\setlength{\fboxsep}{0pt}\fcolorbox[rgb]{1.00,0.00,0.00}{1,1,1}{\strut ##1}}}
\expandafter\def\csname PY@tok@kc\endcsname{\let\PY@bf=\textbf\def\PY@tc##1{\textcolor[rgb]{0.00,0.50,0.00}{##1}}}
\expandafter\def\csname PY@tok@kd\endcsname{\let\PY@bf=\textbf\def\PY@tc##1{\textcolor[rgb]{0.00,0.50,0.00}{##1}}}
\expandafter\def\csname PY@tok@kn\endcsname{\let\PY@bf=\textbf\def\PY@tc##1{\textcolor[rgb]{0.00,0.50,0.00}{##1}}}
\expandafter\def\csname PY@tok@kr\endcsname{\let\PY@bf=\textbf\def\PY@tc##1{\textcolor[rgb]{0.00,0.50,0.00}{##1}}}
\expandafter\def\csname PY@tok@bp\endcsname{\def\PY@tc##1{\textcolor[rgb]{0.00,0.50,0.00}{##1}}}
\expandafter\def\csname PY@tok@fm\endcsname{\def\PY@tc##1{\textcolor[rgb]{0.00,0.00,1.00}{##1}}}
\expandafter\def\csname PY@tok@vc\endcsname{\def\PY@tc##1{\textcolor[rgb]{0.10,0.09,0.49}{##1}}}
\expandafter\def\csname PY@tok@vg\endcsname{\def\PY@tc##1{\textcolor[rgb]{0.10,0.09,0.49}{##1}}}
\expandafter\def\csname PY@tok@vi\endcsname{\def\PY@tc##1{\textcolor[rgb]{0.10,0.09,0.49}{##1}}}
\expandafter\def\csname PY@tok@vm\endcsname{\def\PY@tc##1{\textcolor[rgb]{0.10,0.09,0.49}{##1}}}
\expandafter\def\csname PY@tok@sa\endcsname{\def\PY@tc##1{\textcolor[rgb]{0.73,0.13,0.13}{##1}}}
\expandafter\def\csname PY@tok@sb\endcsname{\def\PY@tc##1{\textcolor[rgb]{0.73,0.13,0.13}{##1}}}
\expandafter\def\csname PY@tok@sc\endcsname{\def\PY@tc##1{\textcolor[rgb]{0.73,0.13,0.13}{##1}}}
\expandafter\def\csname PY@tok@dl\endcsname{\def\PY@tc##1{\textcolor[rgb]{0.73,0.13,0.13}{##1}}}
\expandafter\def\csname PY@tok@s2\endcsname{\def\PY@tc##1{\textcolor[rgb]{0.73,0.13,0.13}{##1}}}
\expandafter\def\csname PY@tok@sh\endcsname{\def\PY@tc##1{\textcolor[rgb]{0.73,0.13,0.13}{##1}}}
\expandafter\def\csname PY@tok@s1\endcsname{\def\PY@tc##1{\textcolor[rgb]{0.73,0.13,0.13}{##1}}}
\expandafter\def\csname PY@tok@mb\endcsname{\def\PY@tc##1{\textcolor[rgb]{0.40,0.40,0.40}{##1}}}
\expandafter\def\csname PY@tok@mf\endcsname{\def\PY@tc##1{\textcolor[rgb]{0.40,0.40,0.40}{##1}}}
\expandafter\def\csname PY@tok@mh\endcsname{\def\PY@tc##1{\textcolor[rgb]{0.40,0.40,0.40}{##1}}}
\expandafter\def\csname PY@tok@mi\endcsname{\def\PY@tc##1{\textcolor[rgb]{0.40,0.40,0.40}{##1}}}
\expandafter\def\csname PY@tok@il\endcsname{\def\PY@tc##1{\textcolor[rgb]{0.40,0.40,0.40}{##1}}}
\expandafter\def\csname PY@tok@mo\endcsname{\def\PY@tc##1{\textcolor[rgb]{0.40,0.40,0.40}{##1}}}
\expandafter\def\csname PY@tok@ch\endcsname{\let\PY@it=\textit\def\PY@tc##1{\textcolor[rgb]{0.25,0.50,0.50}{##1}}}
\expandafter\def\csname PY@tok@cm\endcsname{\let\PY@it=\textit\def\PY@tc##1{\textcolor[rgb]{0.25,0.50,0.50}{##1}}}
\expandafter\def\csname PY@tok@cpf\endcsname{\let\PY@it=\textit\def\PY@tc##1{\textcolor[rgb]{0.25,0.50,0.50}{##1}}}
\expandafter\def\csname PY@tok@c1\endcsname{\let\PY@it=\textit\def\PY@tc##1{\textcolor[rgb]{0.25,0.50,0.50}{##1}}}
\expandafter\def\csname PY@tok@cs\endcsname{\let\PY@it=\textit\def\PY@tc##1{\textcolor[rgb]{0.25,0.50,0.50}{##1}}}

\def\PYZbs{\char`\\}
\def\PYZus{\char`\_}
\def\PYZob{\char`\{}
\def\PYZcb{\char`\}}
\def\PYZca{\char`\^}
\def\PYZam{\char`\&}
\def\PYZlt{\char`\<}
\def\PYZgt{\char`\>}
\def\PYZsh{\char`\#}
\def\PYZpc{\char`\%}
\def\PYZdl{\char`\$}
\def\PYZhy{\char`\-}
\def\PYZsq{\char`\'}
\def\PYZdq{\char`\"}
\def\PYZti{\char`\~}
% for compatibility with earlier versions
\def\PYZat{@}
\def\PYZlb{[}
\def\PYZrb{]}
\makeatother


    % For linebreaks inside Verbatim environment from package fancyvrb. 
    \makeatletter
        \newbox\Wrappedcontinuationbox 
        \newbox\Wrappedvisiblespacebox 
        \newcommand*\Wrappedvisiblespace {\textcolor{red}{\textvisiblespace}} 
        \newcommand*\Wrappedcontinuationsymbol {\textcolor{red}{\llap{\tiny$\m@th\hookrightarrow$}}} 
        \newcommand*\Wrappedcontinuationindent {3ex } 
        \newcommand*\Wrappedafterbreak {\kern\Wrappedcontinuationindent\copy\Wrappedcontinuationbox} 
        % Take advantage of the already applied Pygments mark-up to insert 
        % potential linebreaks for TeX processing. 
        %        {, <, #, %, $, ' and ": go to next line. 
        %        _, }, ^, &, >, - and ~: stay at end of broken line. 
        % Use of \textquotesingle for straight quote. 
        \newcommand*\Wrappedbreaksatspecials {% 
            \def\PYGZus{\discretionary{\char`\_}{\Wrappedafterbreak}{\char`\_}}% 
            \def\PYGZob{\discretionary{}{\Wrappedafterbreak\char`\{}{\char`\{}}% 
            \def\PYGZcb{\discretionary{\char`\}}{\Wrappedafterbreak}{\char`\}}}% 
            \def\PYGZca{\discretionary{\char`\^}{\Wrappedafterbreak}{\char`\^}}% 
            \def\PYGZam{\discretionary{\char`\&}{\Wrappedafterbreak}{\char`\&}}% 
            \def\PYGZlt{\discretionary{}{\Wrappedafterbreak\char`\<}{\char`\<}}% 
            \def\PYGZgt{\discretionary{\char`\>}{\Wrappedafterbreak}{\char`\>}}% 
            \def\PYGZsh{\discretionary{}{\Wrappedafterbreak\char`\#}{\char`\#}}% 
            \def\PYGZpc{\discretionary{}{\Wrappedafterbreak\char`\%}{\char`\%}}% 
            \def\PYGZdl{\discretionary{}{\Wrappedafterbreak\char`\$}{\char`\$}}% 
            \def\PYGZhy{\discretionary{\char`\-}{\Wrappedafterbreak}{\char`\-}}% 
            \def\PYGZsq{\discretionary{}{\Wrappedafterbreak\textquotesingle}{\textquotesingle}}% 
            \def\PYGZdq{\discretionary{}{\Wrappedafterbreak\char`\"}{\char`\"}}% 
            \def\PYGZti{\discretionary{\char`\~}{\Wrappedafterbreak}{\char`\~}}% 
        } 
        % Some characters . , ; ? ! / are not pygmentized. 
        % This macro makes them "active" and they will insert potential linebreaks 
        \newcommand*\Wrappedbreaksatpunct {% 
            \lccode`\~`\.\lowercase{\def~}{\discretionary{\hbox{\char`\.}}{\Wrappedafterbreak}{\hbox{\char`\.}}}% 
            \lccode`\~`\,\lowercase{\def~}{\discretionary{\hbox{\char`\,}}{\Wrappedafterbreak}{\hbox{\char`\,}}}% 
            \lccode`\~`\;\lowercase{\def~}{\discretionary{\hbox{\char`\;}}{\Wrappedafterbreak}{\hbox{\char`\;}}}% 
            \lccode`\~`\:\lowercase{\def~}{\discretionary{\hbox{\char`\:}}{\Wrappedafterbreak}{\hbox{\char`\:}}}% 
            \lccode`\~`\?\lowercase{\def~}{\discretionary{\hbox{\char`\?}}{\Wrappedafterbreak}{\hbox{\char`\?}}}% 
            \lccode`\~`\!\lowercase{\def~}{\discretionary{\hbox{\char`\!}}{\Wrappedafterbreak}{\hbox{\char`\!}}}% 
            \lccode`\~`\/\lowercase{\def~}{\discretionary{\hbox{\char`\/}}{\Wrappedafterbreak}{\hbox{\char`\/}}}% 
            \catcode`\.\active
            \catcode`\,\active 
            \catcode`\;\active
            \catcode`\:\active
            \catcode`\?\active
            \catcode`\!\active
            \catcode`\/\active 
            \lccode`\~`\~ 	
        }
    \makeatother

    \let\OriginalVerbatim=\Verbatim
    \makeatletter
    \renewcommand{\Verbatim}[1][1]{%
        %\parskip\z@skip
        \sbox\Wrappedcontinuationbox {\Wrappedcontinuationsymbol}%
        \sbox\Wrappedvisiblespacebox {\FV@SetupFont\Wrappedvisiblespace}%
        \def\FancyVerbFormatLine ##1{\hsize\linewidth
            \vtop{\raggedright\hyphenpenalty\z@\exhyphenpenalty\z@
                \doublehyphendemerits\z@\finalhyphendemerits\z@
                \strut ##1\strut}%
        }%
        % If the linebreak is at a space, the latter will be displayed as visible
        % space at end of first line, and a continuation symbol starts next line.
        % Stretch/shrink are however usually zero for typewriter font.
        \def\FV@Space {%
            \nobreak\hskip\z@ plus\fontdimen3\font minus\fontdimen4\font
            \discretionary{\copy\Wrappedvisiblespacebox}{\Wrappedafterbreak}
            {\kern\fontdimen2\font}%
        }%
        
        % Allow breaks at special characters using \PYG... macros.
        \Wrappedbreaksatspecials
        % Breaks at punctuation characters . , ; ? ! and / need catcode=\active 	
        \OriginalVerbatim[#1,codes*=\Wrappedbreaksatpunct]%
    }
    \makeatother

    % Exact colors from NB
    \definecolor{incolor}{HTML}{303F9F}
    \definecolor{outcolor}{HTML}{D84315}
    \definecolor{cellborder}{HTML}{CFCFCF}
    \definecolor{cellbackground}{HTML}{F7F7F7}
    
    % prompt
    \makeatletter
    \newcommand{\boxspacing}{\kern\kvtcb@left@rule\kern\kvtcb@boxsep}
    \makeatother
    \newcommand{\prompt}[4]{
        \ttfamily\llap{{\color{#2}[#3]:\hspace{3pt}#4}}\vspace{-\baselineskip}
    }
    

    
    % Prevent overflowing lines due to hard-to-break entities
    \sloppy 
    % Setup hyperref package
    \hypersetup{
      breaklinks=true,  % so long urls are correctly broken across lines
      colorlinks=true,
      urlcolor=urlcolor,
      linkcolor=linkcolor,
      citecolor=citecolor,
      }
    % Slightly bigger margins than the latex defaults
    
    \geometry{verbose,tmargin=1in,bmargin=1in,lmargin=1in,rmargin=1in}
    
    

\begin{document}
    
    \maketitle
    
    

    
    \hypertarget{red-generativa-adversaria}{%
\section{Red Generativa Adversaria}\label{red-generativa-adversaria}}

Creación de números escritos a mano usando MNIST, hecho por Alba
Reinders Sánchez y Alejandro Valverde Mahou, siguiendo el siguiete
\href{https://www.tensorflow.org/tutorials/generative/dcgan}{tutorial}

    \hypertarget{quuxe9-es-una-red-generativa-adversaria}{%
\subsection{¿Qué es una red generativa
adversaria?}\label{quuxe9-es-una-red-generativa-adversaria}}

Las redes generativas adversarias (\emph{GANs}) consisten en dos redes
enfrentadas entre ellas, de forma que la primera intenta generar
imágenes que la segunda reconozca como reales y no como generedas.
Mientras que la segunda intenta lo contrario.

\begin{figure}
\centering
\includegraphics{https://www.tensorflow.org/tutorials/generative/images/gan2.png}
\caption{image.png}
\end{figure}

    Este tutorial consiste en generar imágenes de números escritos a mano
usando el dataset MNIST.

    \hypertarget{imports}{%
\subsection{Imports}\label{imports}}

    \begin{tcolorbox}[breakable, size=fbox, boxrule=1pt, pad at break*=1mm,colback=cellbackground, colframe=cellborder]
\prompt{In}{incolor}{1}{\boxspacing}
\begin{Verbatim}[commandchars=\\\{\}]
\PY{k+kn}{import} \PY{n+nn}{tensorflow} \PY{k}{as} \PY{n+nn}{tf}

\PY{k+kn}{import} \PY{n+nn}{glob}
\PY{k+kn}{import} \PY{n+nn}{imageio}
\PY{k+kn}{import} \PY{n+nn}{matplotlib}\PY{n+nn}{.}\PY{n+nn}{pyplot} \PY{k}{as} \PY{n+nn}{plt}
\PY{k+kn}{import} \PY{n+nn}{numpy} \PY{k}{as} \PY{n+nn}{np}
\PY{k+kn}{import} \PY{n+nn}{os}
\PY{k+kn}{import} \PY{n+nn}{PIL}
\PY{k+kn}{from} \PY{n+nn}{tensorflow}\PY{n+nn}{.}\PY{n+nn}{keras} \PY{k+kn}{import} \PY{n}{layers}
\PY{k+kn}{import} \PY{n+nn}{time}

\PY{k+kn}{from} \PY{n+nn}{IPython} \PY{k+kn}{import} \PY{n}{display}
\end{Verbatim}
\end{tcolorbox}

    \hypertarget{preparaciuxf3n-del-dataset}{%
\subsection{Preparación del dataset}\label{preparaciuxf3n-del-dataset}}

    \begin{tcolorbox}[breakable, size=fbox, boxrule=1pt, pad at break*=1mm,colback=cellbackground, colframe=cellborder]
\prompt{In}{incolor}{2}{\boxspacing}
\begin{Verbatim}[commandchars=\\\{\}]
\PY{n}{BUFFER\PYZus{}SIZE} \PY{o}{=} \PY{l+m+mi}{60000}
\PY{n}{BATCH\PYZus{}SIZE} \PY{o}{=} \PY{l+m+mi}{256}
\end{Verbatim}
\end{tcolorbox}

    \begin{tcolorbox}[breakable, size=fbox, boxrule=1pt, pad at break*=1mm,colback=cellbackground, colframe=cellborder]
\prompt{In}{incolor}{3}{\boxspacing}
\begin{Verbatim}[commandchars=\\\{\}]
\PY{p}{(}\PY{n}{train\PYZus{}images}\PY{p}{,} \PY{n}{train\PYZus{}labels}\PY{p}{)}\PY{p}{,} \PY{p}{(}\PY{n}{\PYZus{}}\PY{p}{,} \PY{n}{\PYZus{}}\PY{p}{)} \PY{o}{=} \PY{n}{tf}\PY{o}{.}\PY{n}{keras}\PY{o}{.}\PY{n}{datasets}\PY{o}{.}\PY{n}{mnist}\PY{o}{.}\PY{n}{load\PYZus{}data}\PY{p}{(}\PY{p}{)}


\PY{n}{train\PYZus{}images} \PY{o}{=} \PY{n}{train\PYZus{}images}\PY{o}{.}\PY{n}{reshape}\PY{p}{(}\PY{n}{train\PYZus{}images}\PY{o}{.}\PY{n}{shape}\PY{p}{[}\PY{l+m+mi}{0}\PY{p}{]}\PY{p}{,} \PY{n}{train\PYZus{}images}\PY{o}{.}\PY{n}{shape}\PY{p}{[}\PY{l+m+mi}{1}\PY{p}{]}\PY{p}{,} \PY{n}{train\PYZus{}images}\PY{o}{.}\PY{n}{shape}\PY{p}{[}\PY{l+m+mi}{2}\PY{p}{]}\PY{p}{,} \PY{l+m+mi}{1}\PY{p}{)}\PY{o}{.}\PY{n}{astype}\PY{p}{(}\PY{l+s+s1}{\PYZsq{}}\PY{l+s+s1}{float32}\PY{l+s+s1}{\PYZsq{}}\PY{p}{)}
\PY{n}{train\PYZus{}images} \PY{o}{=} \PY{p}{(}\PY{n}{train\PYZus{}images} \PY{o}{\PYZhy{}} \PY{l+m+mf}{127.5}\PY{p}{)} \PY{o}{/} \PY{l+m+mf}{127.5} \PY{c+c1}{\PYZsh{} Normalizar las imágener a [\PYZhy{}1, 1]}

\PY{c+c1}{\PYZsh{} Batch y mezcla de los datos}
\PY{n}{train\PYZus{}dataset} \PY{o}{=} \PY{n}{tf}\PY{o}{.}\PY{n}{data}\PY{o}{.}\PY{n}{Dataset}\PY{o}{.}\PY{n}{from\PYZus{}tensor\PYZus{}slices}\PY{p}{(}\PY{n}{train\PYZus{}images}\PY{p}{)}\PY{o}{.}\PY{n}{shuffle}\PY{p}{(}\PY{n}{BUFFER\PYZus{}SIZE}\PY{p}{)}\PY{o}{.}\PY{n}{batch}\PY{p}{(}\PY{n}{BATCH\PYZus{}SIZE}\PY{p}{)}
\end{Verbatim}
\end{tcolorbox}

    \hypertarget{crear-el-modelo}{%
\subsection{Crear el modelo}\label{crear-el-modelo}}

    \hypertarget{modelo-generador}{%
\subsubsection{Modelo Generador}\label{modelo-generador}}

    \begin{tcolorbox}[breakable, size=fbox, boxrule=1pt, pad at break*=1mm,colback=cellbackground, colframe=cellborder]
\prompt{In}{incolor}{4}{\boxspacing}
\begin{Verbatim}[commandchars=\\\{\}]
\PY{k}{def} \PY{n+nf}{make\PYZus{}generator\PYZus{}model}\PY{p}{(}\PY{p}{)}\PY{p}{:}
    \PY{n}{model} \PY{o}{=} \PY{n}{tf}\PY{o}{.}\PY{n}{keras}\PY{o}{.}\PY{n}{Sequential}\PY{p}{(}\PY{p}{)}
    \PY{n}{model}\PY{o}{.}\PY{n}{add}\PY{p}{(}\PY{n}{layers}\PY{o}{.}\PY{n}{Dense}\PY{p}{(}\PY{l+m+mi}{7}\PY{o}{*}\PY{l+m+mi}{7}\PY{o}{*}\PY{l+m+mi}{256}\PY{p}{,} \PY{n}{use\PYZus{}bias}\PY{o}{=}\PY{k+kc}{False}\PY{p}{,} \PY{n}{input\PYZus{}shape}\PY{o}{=}\PY{p}{(}\PY{l+m+mi}{100}\PY{p}{,}\PY{p}{)}\PY{p}{)}\PY{p}{)}
    \PY{n}{model}\PY{o}{.}\PY{n}{add}\PY{p}{(}\PY{n}{layers}\PY{o}{.}\PY{n}{BatchNormalization}\PY{p}{(}\PY{p}{)}\PY{p}{)}
    \PY{n}{model}\PY{o}{.}\PY{n}{add}\PY{p}{(}\PY{n}{layers}\PY{o}{.}\PY{n}{LeakyReLU}\PY{p}{(}\PY{p}{)}\PY{p}{)}

    \PY{n}{model}\PY{o}{.}\PY{n}{add}\PY{p}{(}\PY{n}{layers}\PY{o}{.}\PY{n}{Reshape}\PY{p}{(}\PY{p}{(}\PY{l+m+mi}{7}\PY{p}{,} \PY{l+m+mi}{7}\PY{p}{,} \PY{l+m+mi}{256}\PY{p}{)}\PY{p}{)}\PY{p}{)}
    \PY{k}{assert} \PY{n}{model}\PY{o}{.}\PY{n}{output\PYZus{}shape} \PY{o}{==} \PY{p}{(}\PY{k+kc}{None}\PY{p}{,} \PY{l+m+mi}{7}\PY{p}{,} \PY{l+m+mi}{7}\PY{p}{,} \PY{l+m+mi}{256}\PY{p}{)} 

    \PY{n}{model}\PY{o}{.}\PY{n}{add}\PY{p}{(}\PY{n}{layers}\PY{o}{.}\PY{n}{Conv2DTranspose}\PY{p}{(}\PY{l+m+mi}{128}\PY{p}{,} \PY{p}{(}\PY{l+m+mi}{5}\PY{p}{,} \PY{l+m+mi}{5}\PY{p}{)}\PY{p}{,} \PY{n}{strides}\PY{o}{=}\PY{p}{(}\PY{l+m+mi}{1}\PY{p}{,} \PY{l+m+mi}{1}\PY{p}{)}\PY{p}{,} \PY{n}{padding}\PY{o}{=}\PY{l+s+s1}{\PYZsq{}}\PY{l+s+s1}{same}\PY{l+s+s1}{\PYZsq{}}\PY{p}{,} \PY{n}{use\PYZus{}bias}\PY{o}{=}\PY{k+kc}{False}\PY{p}{)}\PY{p}{)}
    \PY{k}{assert} \PY{n}{model}\PY{o}{.}\PY{n}{output\PYZus{}shape} \PY{o}{==} \PY{p}{(}\PY{k+kc}{None}\PY{p}{,} \PY{l+m+mi}{7}\PY{p}{,} \PY{l+m+mi}{7}\PY{p}{,} \PY{l+m+mi}{128}\PY{p}{)}
    \PY{n}{model}\PY{o}{.}\PY{n}{add}\PY{p}{(}\PY{n}{layers}\PY{o}{.}\PY{n}{BatchNormalization}\PY{p}{(}\PY{p}{)}\PY{p}{)}
    \PY{n}{model}\PY{o}{.}\PY{n}{add}\PY{p}{(}\PY{n}{layers}\PY{o}{.}\PY{n}{LeakyReLU}\PY{p}{(}\PY{p}{)}\PY{p}{)}

    \PY{n}{model}\PY{o}{.}\PY{n}{add}\PY{p}{(}\PY{n}{layers}\PY{o}{.}\PY{n}{Conv2DTranspose}\PY{p}{(}\PY{l+m+mi}{64}\PY{p}{,} \PY{p}{(}\PY{l+m+mi}{5}\PY{p}{,} \PY{l+m+mi}{5}\PY{p}{)}\PY{p}{,} \PY{n}{strides}\PY{o}{=}\PY{p}{(}\PY{l+m+mi}{2}\PY{p}{,} \PY{l+m+mi}{2}\PY{p}{)}\PY{p}{,} \PY{n}{padding}\PY{o}{=}\PY{l+s+s1}{\PYZsq{}}\PY{l+s+s1}{same}\PY{l+s+s1}{\PYZsq{}}\PY{p}{,} \PY{n}{use\PYZus{}bias}\PY{o}{=}\PY{k+kc}{False}\PY{p}{)}\PY{p}{)}
    \PY{k}{assert} \PY{n}{model}\PY{o}{.}\PY{n}{output\PYZus{}shape} \PY{o}{==} \PY{p}{(}\PY{k+kc}{None}\PY{p}{,} \PY{l+m+mi}{14}\PY{p}{,} \PY{l+m+mi}{14}\PY{p}{,} \PY{l+m+mi}{64}\PY{p}{)}
    \PY{n}{model}\PY{o}{.}\PY{n}{add}\PY{p}{(}\PY{n}{layers}\PY{o}{.}\PY{n}{BatchNormalization}\PY{p}{(}\PY{p}{)}\PY{p}{)}
    \PY{n}{model}\PY{o}{.}\PY{n}{add}\PY{p}{(}\PY{n}{layers}\PY{o}{.}\PY{n}{LeakyReLU}\PY{p}{(}\PY{p}{)}\PY{p}{)}

    \PY{n}{model}\PY{o}{.}\PY{n}{add}\PY{p}{(}\PY{n}{layers}\PY{o}{.}\PY{n}{Conv2DTranspose}\PY{p}{(}\PY{l+m+mi}{1}\PY{p}{,} \PY{p}{(}\PY{l+m+mi}{5}\PY{p}{,} \PY{l+m+mi}{5}\PY{p}{)}\PY{p}{,} \PY{n}{strides}\PY{o}{=}\PY{p}{(}\PY{l+m+mi}{2}\PY{p}{,} \PY{l+m+mi}{2}\PY{p}{)}\PY{p}{,} \PY{n}{padding}\PY{o}{=}\PY{l+s+s1}{\PYZsq{}}\PY{l+s+s1}{same}\PY{l+s+s1}{\PYZsq{}}\PY{p}{,} \PY{n}{use\PYZus{}bias}\PY{o}{=}\PY{k+kc}{False}\PY{p}{,} \PY{n}{activation}\PY{o}{=}\PY{l+s+s1}{\PYZsq{}}\PY{l+s+s1}{tanh}\PY{l+s+s1}{\PYZsq{}}\PY{p}{)}\PY{p}{)}
    \PY{k}{assert} \PY{n}{model}\PY{o}{.}\PY{n}{output\PYZus{}shape} \PY{o}{==} \PY{p}{(}\PY{k+kc}{None}\PY{p}{,} \PY{l+m+mi}{28}\PY{p}{,} \PY{l+m+mi}{28}\PY{p}{,} \PY{l+m+mi}{1}\PY{p}{)}

    \PY{k}{return} \PY{n}{model}
\end{Verbatim}
\end{tcolorbox}

    \hypertarget{generar-imagen-sin-entrenar-el-modelo}{%
\paragraph{Generar imagen sin entrenar el
modelo}\label{generar-imagen-sin-entrenar-el-modelo}}

    \begin{tcolorbox}[breakable, size=fbox, boxrule=1pt, pad at break*=1mm,colback=cellbackground, colframe=cellborder]
\prompt{In}{incolor}{5}{\boxspacing}
\begin{Verbatim}[commandchars=\\\{\}]
\PY{n}{generator} \PY{o}{=} \PY{n}{make\PYZus{}generator\PYZus{}model}\PY{p}{(}\PY{p}{)}

\PY{n}{noise} \PY{o}{=} \PY{n}{tf}\PY{o}{.}\PY{n}{random}\PY{o}{.}\PY{n}{normal}\PY{p}{(}\PY{p}{[}\PY{l+m+mi}{1}\PY{p}{,} \PY{l+m+mi}{100}\PY{p}{]}\PY{p}{)}
\PY{n}{generated\PYZus{}image} \PY{o}{=} \PY{n}{generator}\PY{p}{(}\PY{n}{noise}\PY{p}{,} \PY{n}{training}\PY{o}{=}\PY{k+kc}{False}\PY{p}{)}

\PY{n}{plt}\PY{o}{.}\PY{n}{imshow}\PY{p}{(}\PY{n}{generated\PYZus{}image}\PY{p}{[}\PY{l+m+mi}{0}\PY{p}{,} \PY{p}{:}\PY{p}{,} \PY{p}{:}\PY{p}{,} \PY{l+m+mi}{0}\PY{p}{]}\PY{p}{,} \PY{n}{cmap}\PY{o}{=}\PY{l+s+s1}{\PYZsq{}}\PY{l+s+s1}{gray}\PY{l+s+s1}{\PYZsq{}}\PY{p}{)}
\end{Verbatim}
\end{tcolorbox}

            \begin{tcolorbox}[breakable, size=fbox, boxrule=.5pt, pad at break*=1mm, opacityfill=0]
\prompt{Out}{outcolor}{5}{\boxspacing}
\begin{Verbatim}[commandchars=\\\{\}]
<matplotlib.image.AxesImage at 0x7f76d44d9bb0>
\end{Verbatim}
\end{tcolorbox}
        
    \begin{center}
    \adjustimage{max size={0.9\linewidth}{0.9\paperheight}}{output_12_1.png}
    \end{center}
    { \hspace*{\fill} \\}
    
    \hypertarget{modelo-discriminador}{%
\subsubsection{Modelo Discriminador}\label{modelo-discriminador}}

    \begin{tcolorbox}[breakable, size=fbox, boxrule=1pt, pad at break*=1mm,colback=cellbackground, colframe=cellborder]
\prompt{In}{incolor}{6}{\boxspacing}
\begin{Verbatim}[commandchars=\\\{\}]
\PY{k}{def} \PY{n+nf}{make\PYZus{}discriminator\PYZus{}model}\PY{p}{(}\PY{p}{)}\PY{p}{:}
    \PY{n}{model} \PY{o}{=} \PY{n}{tf}\PY{o}{.}\PY{n}{keras}\PY{o}{.}\PY{n}{Sequential}\PY{p}{(}\PY{p}{)}
    \PY{n}{model}\PY{o}{.}\PY{n}{add}\PY{p}{(}\PY{n}{layers}\PY{o}{.}\PY{n}{Conv2D}\PY{p}{(}\PY{l+m+mi}{64}\PY{p}{,} \PY{p}{(}\PY{l+m+mi}{5}\PY{p}{,} \PY{l+m+mi}{5}\PY{p}{)}\PY{p}{,} \PY{n}{strides}\PY{o}{=}\PY{p}{(}\PY{l+m+mi}{2}\PY{p}{,} \PY{l+m+mi}{2}\PY{p}{)}\PY{p}{,} \PY{n}{padding}\PY{o}{=}\PY{l+s+s1}{\PYZsq{}}\PY{l+s+s1}{same}\PY{l+s+s1}{\PYZsq{}}\PY{p}{,} \PY{n}{input\PYZus{}shape}\PY{o}{=}\PY{p}{[}\PY{l+m+mi}{28}\PY{p}{,} \PY{l+m+mi}{28}\PY{p}{,} \PY{l+m+mi}{1}\PY{p}{]}\PY{p}{)}\PY{p}{)}
    \PY{n}{model}\PY{o}{.}\PY{n}{add}\PY{p}{(}\PY{n}{layers}\PY{o}{.}\PY{n}{LeakyReLU}\PY{p}{(}\PY{p}{)}\PY{p}{)}
    \PY{n}{model}\PY{o}{.}\PY{n}{add}\PY{p}{(}\PY{n}{layers}\PY{o}{.}\PY{n}{Dropout}\PY{p}{(}\PY{l+m+mf}{0.3}\PY{p}{)}\PY{p}{)}

    \PY{n}{model}\PY{o}{.}\PY{n}{add}\PY{p}{(}\PY{n}{layers}\PY{o}{.}\PY{n}{Conv2D}\PY{p}{(}\PY{l+m+mi}{128}\PY{p}{,} \PY{p}{(}\PY{l+m+mi}{5}\PY{p}{,} \PY{l+m+mi}{5}\PY{p}{)}\PY{p}{,} \PY{n}{strides}\PY{o}{=}\PY{p}{(}\PY{l+m+mi}{2}\PY{p}{,} \PY{l+m+mi}{2}\PY{p}{)}\PY{p}{,} \PY{n}{padding}\PY{o}{=}\PY{l+s+s1}{\PYZsq{}}\PY{l+s+s1}{same}\PY{l+s+s1}{\PYZsq{}}\PY{p}{)}\PY{p}{)}
    \PY{n}{model}\PY{o}{.}\PY{n}{add}\PY{p}{(}\PY{n}{layers}\PY{o}{.}\PY{n}{LeakyReLU}\PY{p}{(}\PY{p}{)}\PY{p}{)}
    \PY{n}{model}\PY{o}{.}\PY{n}{add}\PY{p}{(}\PY{n}{layers}\PY{o}{.}\PY{n}{Dropout}\PY{p}{(}\PY{l+m+mf}{0.3}\PY{p}{)}\PY{p}{)}

    \PY{n}{model}\PY{o}{.}\PY{n}{add}\PY{p}{(}\PY{n}{layers}\PY{o}{.}\PY{n}{Flatten}\PY{p}{(}\PY{p}{)}\PY{p}{)}
    \PY{n}{model}\PY{o}{.}\PY{n}{add}\PY{p}{(}\PY{n}{layers}\PY{o}{.}\PY{n}{Dense}\PY{p}{(}\PY{l+m+mi}{1}\PY{p}{)}\PY{p}{)}

    \PY{k}{return} \PY{n}{model}
\end{Verbatim}
\end{tcolorbox}

    \begin{tcolorbox}[breakable, size=fbox, boxrule=1pt, pad at break*=1mm,colback=cellbackground, colframe=cellborder]
\prompt{In}{incolor}{7}{\boxspacing}
\begin{Verbatim}[commandchars=\\\{\}]
\PY{n}{discriminator} \PY{o}{=} \PY{n}{make\PYZus{}discriminator\PYZus{}model}\PY{p}{(}\PY{p}{)}
\PY{n}{decision} \PY{o}{=} \PY{n}{discriminator}\PY{p}{(}\PY{n}{generated\PYZus{}image}\PY{p}{)}
\PY{n+nb}{print} \PY{p}{(}\PY{n}{decision}\PY{p}{)}
\end{Verbatim}
\end{tcolorbox}

    \begin{Verbatim}[commandchars=\\\{\}]
tf.Tensor([[0.00065996]], shape=(1, 1), dtype=float32)
    \end{Verbatim}

    \hypertarget{definir-las-funciones-de-coste-y-los-optimizadores}{%
\subsection{Definir las funciones de coste y los
optimizadores}\label{definir-las-funciones-de-coste-y-los-optimizadores}}

    \begin{tcolorbox}[breakable, size=fbox, boxrule=1pt, pad at break*=1mm,colback=cellbackground, colframe=cellborder]
\prompt{In}{incolor}{8}{\boxspacing}
\begin{Verbatim}[commandchars=\\\{\}]
\PY{c+c1}{\PYZsh{} Función auxiliar para la \PYZsq{}cross entropy\PYZsq{}}
\PY{n}{cross\PYZus{}entropy} \PY{o}{=} \PY{n}{tf}\PY{o}{.}\PY{n}{keras}\PY{o}{.}\PY{n}{losses}\PY{o}{.}\PY{n}{BinaryCrossentropy}\PY{p}{(}\PY{n}{from\PYZus{}logits}\PY{o}{=}\PY{k+kc}{True}\PY{p}{)}
\end{Verbatim}
\end{tcolorbox}

    \hypertarget{funciuxf3n-de-coste-del-discriminador}{%
\subsubsection{Función de coste del
discriminador}\label{funciuxf3n-de-coste-del-discriminador}}

Este método cuantifica cómo de bien es capaz el discriminador de
distinguir entre imágenes reales y generadas. Para ello, compara la
salida de la predicción de las imágenes reales con un array de 1s y la
predicción de imágenes generadas con un array de 0s, y lo suma todo.

    \begin{tcolorbox}[breakable, size=fbox, boxrule=1pt, pad at break*=1mm,colback=cellbackground, colframe=cellborder]
\prompt{In}{incolor}{9}{\boxspacing}
\begin{Verbatim}[commandchars=\\\{\}]
\PY{k}{def} \PY{n+nf}{discriminator\PYZus{}loss}\PY{p}{(}\PY{n}{real\PYZus{}output}\PY{p}{,} \PY{n}{fake\PYZus{}output}\PY{p}{)}\PY{p}{:}
    \PY{n}{real\PYZus{}loss} \PY{o}{=} \PY{n}{cross\PYZus{}entropy}\PY{p}{(}\PY{n}{tf}\PY{o}{.}\PY{n}{ones\PYZus{}like}\PY{p}{(}\PY{n}{real\PYZus{}output}\PY{p}{)}\PY{p}{,} \PY{n}{real\PYZus{}output}\PY{p}{)}
    \PY{n}{fake\PYZus{}loss} \PY{o}{=} \PY{n}{cross\PYZus{}entropy}\PY{p}{(}\PY{n}{tf}\PY{o}{.}\PY{n}{zeros\PYZus{}like}\PY{p}{(}\PY{n}{fake\PYZus{}output}\PY{p}{)}\PY{p}{,} \PY{n}{fake\PYZus{}output}\PY{p}{)}
    \PY{n}{total\PYZus{}loss} \PY{o}{=} \PY{n}{real\PYZus{}loss} \PY{o}{+} \PY{n}{fake\PYZus{}loss}
    \PY{k}{return} \PY{n}{total\PYZus{}loss}
\end{Verbatim}
\end{tcolorbox}

    \hypertarget{funciuxf3n-de-coste-del-generador}{%
\subsubsection{Función de coste del
generador}\label{funciuxf3n-de-coste-del-generador}}

Este método cuantifica si el generador es capaz de engañar al
discriminador. Si el generador lo está haciendo bien, el discriminador
clasificará las imágenes generadas como reales (1s).

    \begin{tcolorbox}[breakable, size=fbox, boxrule=1pt, pad at break*=1mm,colback=cellbackground, colframe=cellborder]
\prompt{In}{incolor}{10}{\boxspacing}
\begin{Verbatim}[commandchars=\\\{\}]
\PY{k}{def} \PY{n+nf}{generator\PYZus{}loss}\PY{p}{(}\PY{n}{fake\PYZus{}output}\PY{p}{)}\PY{p}{:}
    \PY{k}{return} \PY{n}{cross\PYZus{}entropy}\PY{p}{(}\PY{n}{tf}\PY{o}{.}\PY{n}{ones\PYZus{}like}\PY{p}{(}\PY{n}{fake\PYZus{}output}\PY{p}{)}\PY{p}{,} \PY{n}{fake\PYZus{}output}\PY{p}{)}
\end{Verbatim}
\end{tcolorbox}

    \hypertarget{optimizadores}{%
\subsubsection{Optimizadores}\label{optimizadores}}

Cada red tiene que tener su optimizador.

    \begin{tcolorbox}[breakable, size=fbox, boxrule=1pt, pad at break*=1mm,colback=cellbackground, colframe=cellborder]
\prompt{In}{incolor}{11}{\boxspacing}
\begin{Verbatim}[commandchars=\\\{\}]
\PY{n}{generator\PYZus{}optimizer} \PY{o}{=} \PY{n}{tf}\PY{o}{.}\PY{n}{keras}\PY{o}{.}\PY{n}{optimizers}\PY{o}{.}\PY{n}{Adam}\PY{p}{(}\PY{l+m+mf}{1e\PYZhy{}4}\PY{p}{)}
\PY{n}{discriminator\PYZus{}optimizer} \PY{o}{=} \PY{n}{tf}\PY{o}{.}\PY{n}{keras}\PY{o}{.}\PY{n}{optimizers}\PY{o}{.}\PY{n}{Adam}\PY{p}{(}\PY{l+m+mf}{1e\PYZhy{}4}\PY{p}{)}
\end{Verbatim}
\end{tcolorbox}

    \hypertarget{checkpoints}{%
\subsubsection{Checkpoints}\label{checkpoints}}

Los checkpoints son necesarios para guardar el progreso del
entrenamiento y poder restaurarlo en caso de que se vea interrumpido.

    \begin{tcolorbox}[breakable, size=fbox, boxrule=1pt, pad at break*=1mm,colback=cellbackground, colframe=cellborder]
\prompt{In}{incolor}{12}{\boxspacing}
\begin{Verbatim}[commandchars=\\\{\}]
\PY{n}{checkpoint\PYZus{}dir} \PY{o}{=} \PY{l+s+s1}{\PYZsq{}}\PY{l+s+s1}{./training\PYZus{}checkpoints}\PY{l+s+s1}{\PYZsq{}}
\PY{n}{checkpoint\PYZus{}prefix} \PY{o}{=} \PY{n}{os}\PY{o}{.}\PY{n}{path}\PY{o}{.}\PY{n}{join}\PY{p}{(}\PY{n}{checkpoint\PYZus{}dir}\PY{p}{,} \PY{l+s+s2}{\PYZdq{}}\PY{l+s+s2}{ckpt}\PY{l+s+s2}{\PYZdq{}}\PY{p}{)}
\PY{n}{checkpoint} \PY{o}{=} \PY{n}{tf}\PY{o}{.}\PY{n}{train}\PY{o}{.}\PY{n}{Checkpoint}\PY{p}{(}\PY{n}{generator\PYZus{}optimizer}\PY{o}{=}\PY{n}{generator\PYZus{}optimizer}\PY{p}{,}
                                 \PY{n}{discriminator\PYZus{}optimizer}\PY{o}{=}\PY{n}{discriminator\PYZus{}optimizer}\PY{p}{,}
                                 \PY{n}{generator}\PY{o}{=}\PY{n}{generator}\PY{p}{,}
                                 \PY{n}{discriminator}\PY{o}{=}\PY{n}{discriminator}\PY{p}{)}
\end{Verbatim}
\end{tcolorbox}

    \hypertarget{bucle-de-entrenamiento}{%
\subsection{Bucle de entrenamiento}\label{bucle-de-entrenamiento}}

    \begin{tcolorbox}[breakable, size=fbox, boxrule=1pt, pad at break*=1mm,colback=cellbackground, colframe=cellborder]
\prompt{In}{incolor}{13}{\boxspacing}
\begin{Verbatim}[commandchars=\\\{\}]
\PY{n}{EPOCHS} \PY{o}{=} \PY{l+m+mi}{50}
\PY{n}{noise\PYZus{}dim} \PY{o}{=} \PY{l+m+mi}{100}
\PY{n}{num\PYZus{}examples\PYZus{}to\PYZus{}generate} \PY{o}{=} \PY{l+m+mi}{16}

\PY{n}{seed} \PY{o}{=} \PY{n}{tf}\PY{o}{.}\PY{n}{random}\PY{o}{.}\PY{n}{normal}\PY{p}{(}\PY{p}{[}\PY{n}{num\PYZus{}examples\PYZus{}to\PYZus{}generate}\PY{p}{,} \PY{n}{noise\PYZus{}dim}\PY{p}{]}\PY{p}{)}
\end{Verbatim}
\end{tcolorbox}

    El bucle de entrenamiento comienza con el generador recibiendo una
`seed' como input. Esa `seed' se utiliza para generar una imagen.
Después el discriminador se usa para clasificar imágenes reales e
imágenes falsas.

Se calcula la función de coste para cada uno de los modelos y se
utilizan los gradientes para actualizar ambos modelos.

    \begin{tcolorbox}[breakable, size=fbox, boxrule=1pt, pad at break*=1mm,colback=cellbackground, colframe=cellborder]
\prompt{In}{incolor}{14}{\boxspacing}
\begin{Verbatim}[commandchars=\\\{\}]
\PY{n+nd}{@tf}\PY{o}{.}\PY{n}{function}
\PY{k}{def} \PY{n+nf}{train\PYZus{}step}\PY{p}{(}\PY{n}{images}\PY{p}{)}\PY{p}{:}
    \PY{n}{noise} \PY{o}{=} \PY{n}{tf}\PY{o}{.}\PY{n}{random}\PY{o}{.}\PY{n}{normal}\PY{p}{(}\PY{p}{[}\PY{n}{BATCH\PYZus{}SIZE}\PY{p}{,} \PY{n}{noise\PYZus{}dim}\PY{p}{]}\PY{p}{)}

    \PY{k}{with} \PY{n}{tf}\PY{o}{.}\PY{n}{GradientTape}\PY{p}{(}\PY{p}{)} \PY{k}{as} \PY{n}{gen\PYZus{}tape}\PY{p}{,} \PY{n}{tf}\PY{o}{.}\PY{n}{GradientTape}\PY{p}{(}\PY{p}{)} \PY{k}{as} \PY{n}{disc\PYZus{}tape}\PY{p}{:}
        \PY{n}{generated\PYZus{}images} \PY{o}{=} \PY{n}{generator}\PY{p}{(}\PY{n}{noise}\PY{p}{,} \PY{n}{training}\PY{o}{=}\PY{k+kc}{True}\PY{p}{)}

        \PY{n}{real\PYZus{}output} \PY{o}{=} \PY{n}{discriminator}\PY{p}{(}\PY{n}{images}\PY{p}{,} \PY{n}{training}\PY{o}{=}\PY{k+kc}{True}\PY{p}{)}
        \PY{n}{fake\PYZus{}output} \PY{o}{=} \PY{n}{discriminator}\PY{p}{(}\PY{n}{generated\PYZus{}images}\PY{p}{,} \PY{n}{training}\PY{o}{=}\PY{k+kc}{True}\PY{p}{)}

        \PY{n}{gen\PYZus{}loss} \PY{o}{=} \PY{n}{generator\PYZus{}loss}\PY{p}{(}\PY{n}{fake\PYZus{}output}\PY{p}{)}
        \PY{n}{disc\PYZus{}loss} \PY{o}{=} \PY{n}{discriminator\PYZus{}loss}\PY{p}{(}\PY{n}{real\PYZus{}output}\PY{p}{,} \PY{n}{fake\PYZus{}output}\PY{p}{)}

    \PY{n}{gradients\PYZus{}of\PYZus{}generator} \PY{o}{=} \PY{n}{gen\PYZus{}tape}\PY{o}{.}\PY{n}{gradient}\PY{p}{(}\PY{n}{gen\PYZus{}loss}\PY{p}{,} \PY{n}{generator}\PY{o}{.}\PY{n}{trainable\PYZus{}variables}\PY{p}{)}
    \PY{n}{gradients\PYZus{}of\PYZus{}discriminator} \PY{o}{=} \PY{n}{disc\PYZus{}tape}\PY{o}{.}\PY{n}{gradient}\PY{p}{(}\PY{n}{disc\PYZus{}loss}\PY{p}{,} \PY{n}{discriminator}\PY{o}{.}\PY{n}{trainable\PYZus{}variables}\PY{p}{)}

    \PY{n}{generator\PYZus{}optimizer}\PY{o}{.}\PY{n}{apply\PYZus{}gradients}\PY{p}{(}\PY{n+nb}{zip}\PY{p}{(}\PY{n}{gradients\PYZus{}of\PYZus{}generator}\PY{p}{,} \PY{n}{generator}\PY{o}{.}\PY{n}{trainable\PYZus{}variables}\PY{p}{)}\PY{p}{)}
    \PY{n}{discriminator\PYZus{}optimizer}\PY{o}{.}\PY{n}{apply\PYZus{}gradients}\PY{p}{(}\PY{n+nb}{zip}\PY{p}{(}\PY{n}{gradients\PYZus{}of\PYZus{}discriminator}\PY{p}{,} \PY{n}{discriminator}\PY{o}{.}\PY{n}{trainable\PYZus{}variables}\PY{p}{)}\PY{p}{)}
\end{Verbatim}
\end{tcolorbox}

    \begin{tcolorbox}[breakable, size=fbox, boxrule=1pt, pad at break*=1mm,colback=cellbackground, colframe=cellborder]
\prompt{In}{incolor}{15}{\boxspacing}
\begin{Verbatim}[commandchars=\\\{\}]
\PY{k}{def} \PY{n+nf}{train}\PY{p}{(}\PY{n}{dataset}\PY{p}{,} \PY{n}{epochs}\PY{p}{)}\PY{p}{:}
    \PY{k}{for} \PY{n}{epoch} \PY{o+ow}{in} \PY{n+nb}{range}\PY{p}{(}\PY{n}{epochs}\PY{p}{)}\PY{p}{:}
        \PY{n}{start} \PY{o}{=} \PY{n}{time}\PY{o}{.}\PY{n}{time}\PY{p}{(}\PY{p}{)}

        \PY{k}{for} \PY{n}{image\PYZus{}batch} \PY{o+ow}{in} \PY{n}{dataset}\PY{p}{:}
            \PY{n}{train\PYZus{}step}\PY{p}{(}\PY{n}{image\PYZus{}batch}\PY{p}{)}

        \PY{c+c1}{\PYZsh{} Producir imágenes para el GIF}
        \PY{n}{display}\PY{o}{.}\PY{n}{clear\PYZus{}output}\PY{p}{(}\PY{n}{wait}\PY{o}{=}\PY{k+kc}{True}\PY{p}{)}
        \PY{n}{generate\PYZus{}and\PYZus{}save\PYZus{}images}\PY{p}{(}\PY{n}{generator}\PY{p}{,}
                                 \PY{n}{epoch} \PY{o}{+} \PY{l+m+mi}{1}\PY{p}{,}
                                 \PY{n}{seed}\PY{p}{)}

        \PY{c+c1}{\PYZsh{} Guardar el modelo cada 15 epochs}
        \PY{k}{if} \PY{p}{(}\PY{n}{epoch} \PY{o}{+} \PY{l+m+mi}{1}\PY{p}{)} \PY{o}{\PYZpc{}} \PY{l+m+mi}{15} \PY{o}{==} \PY{l+m+mi}{0}\PY{p}{:}
            \PY{n}{checkpoint}\PY{o}{.}\PY{n}{save}\PY{p}{(}\PY{n}{file\PYZus{}prefix} \PY{o}{=} \PY{n}{checkpoint\PYZus{}prefix}\PY{p}{)}

        \PY{n+nb}{print} \PY{p}{(}\PY{l+s+s1}{\PYZsq{}}\PY{l+s+s1}{Time for epoch }\PY{l+s+si}{\PYZob{}\PYZcb{}}\PY{l+s+s1}{ is }\PY{l+s+si}{\PYZob{}\PYZcb{}}\PY{l+s+s1}{ sec}\PY{l+s+s1}{\PYZsq{}}\PY{o}{.}\PY{n}{format}\PY{p}{(}\PY{n}{epoch} \PY{o}{+} \PY{l+m+mi}{1}\PY{p}{,} \PY{n}{time}\PY{o}{.}\PY{n}{time}\PY{p}{(}\PY{p}{)}\PY{o}{\PYZhy{}}\PY{n}{start}\PY{p}{)}\PY{p}{)}

    \PY{c+c1}{\PYZsh{} Generar después del último epoch}
    \PY{n}{display}\PY{o}{.}\PY{n}{clear\PYZus{}output}\PY{p}{(}\PY{n}{wait}\PY{o}{=}\PY{k+kc}{True}\PY{p}{)}
    \PY{n}{generate\PYZus{}and\PYZus{}save\PYZus{}images}\PY{p}{(}\PY{n}{generator}\PY{p}{,}
                               \PY{n}{epochs}\PY{p}{,}
                               \PY{n}{seed}\PY{p}{)}
\end{Verbatim}
\end{tcolorbox}

    \begin{tcolorbox}[breakable, size=fbox, boxrule=1pt, pad at break*=1mm,colback=cellbackground, colframe=cellborder]
\prompt{In}{incolor}{16}{\boxspacing}
\begin{Verbatim}[commandchars=\\\{\}]
\PY{c+c1}{\PYZsh{} Generar y guardar imágenes}
\PY{k}{def} \PY{n+nf}{generate\PYZus{}and\PYZus{}save\PYZus{}images}\PY{p}{(}\PY{n}{model}\PY{p}{,} \PY{n}{epoch}\PY{p}{,} \PY{n}{test\PYZus{}input}\PY{p}{)}\PY{p}{:}
    
    \PY{n}{predictions} \PY{o}{=} \PY{n}{model}\PY{p}{(}\PY{n}{test\PYZus{}input}\PY{p}{,} \PY{n}{training}\PY{o}{=}\PY{k+kc}{False}\PY{p}{)}

    \PY{n}{fig} \PY{o}{=} \PY{n}{plt}\PY{o}{.}\PY{n}{figure}\PY{p}{(}\PY{n}{figsize}\PY{o}{=}\PY{p}{(}\PY{l+m+mi}{4}\PY{p}{,}\PY{l+m+mi}{4}\PY{p}{)}\PY{p}{)}

    \PY{k}{for} \PY{n}{i} \PY{o+ow}{in} \PY{n+nb}{range}\PY{p}{(}\PY{n}{predictions}\PY{o}{.}\PY{n}{shape}\PY{p}{[}\PY{l+m+mi}{0}\PY{p}{]}\PY{p}{)}\PY{p}{:}
        \PY{n}{plt}\PY{o}{.}\PY{n}{subplot}\PY{p}{(}\PY{l+m+mi}{4}\PY{p}{,} \PY{l+m+mi}{4}\PY{p}{,} \PY{n}{i}\PY{o}{+}\PY{l+m+mi}{1}\PY{p}{)}
        \PY{n}{plt}\PY{o}{.}\PY{n}{imshow}\PY{p}{(}\PY{n}{predictions}\PY{p}{[}\PY{n}{i}\PY{p}{,} \PY{p}{:}\PY{p}{,} \PY{p}{:}\PY{p}{,} \PY{l+m+mi}{0}\PY{p}{]} \PY{o}{*} \PY{l+m+mf}{127.5} \PY{o}{+} \PY{l+m+mf}{127.5}\PY{p}{,} \PY{n}{cmap}\PY{o}{=}\PY{l+s+s1}{\PYZsq{}}\PY{l+s+s1}{gray}\PY{l+s+s1}{\PYZsq{}}\PY{p}{)}
        \PY{n}{plt}\PY{o}{.}\PY{n}{axis}\PY{p}{(}\PY{l+s+s1}{\PYZsq{}}\PY{l+s+s1}{off}\PY{l+s+s1}{\PYZsq{}}\PY{p}{)}

    \PY{n}{plt}\PY{o}{.}\PY{n}{savefig}\PY{p}{(}\PY{l+s+s1}{\PYZsq{}}\PY{l+s+s1}{image\PYZus{}at\PYZus{}epoch\PYZus{}}\PY{l+s+si}{\PYZob{}:04d\PYZcb{}}\PY{l+s+s1}{.png}\PY{l+s+s1}{\PYZsq{}}\PY{o}{.}\PY{n}{format}\PY{p}{(}\PY{n}{epoch}\PY{p}{)}\PY{p}{)}
    \PY{n}{plt}\PY{o}{.}\PY{n}{show}\PY{p}{(}\PY{p}{)}
\end{Verbatim}
\end{tcolorbox}

    \hypertarget{entrenar-el-modelo}{%
\subsection{Entrenar el modelo}\label{entrenar-el-modelo}}

    \begin{tcolorbox}[breakable, size=fbox, boxrule=1pt, pad at break*=1mm,colback=cellbackground, colframe=cellborder]
\prompt{In}{incolor}{17}{\boxspacing}
\begin{Verbatim}[commandchars=\\\{\}]
\PY{n}{train}\PY{p}{(}\PY{n}{train\PYZus{}dataset}\PY{p}{,} \PY{n}{EPOCHS}\PY{p}{)}
\end{Verbatim}
\end{tcolorbox}

    \begin{center}
    \adjustimage{max size={0.9\linewidth}{0.9\paperheight}}{output_33_0.png}
    \end{center}
    { \hspace*{\fill} \\}
    
    \begin{tcolorbox}[breakable, size=fbox, boxrule=1pt, pad at break*=1mm,colback=cellbackground, colframe=cellborder]
\prompt{In}{incolor}{18}{\boxspacing}
\begin{Verbatim}[commandchars=\\\{\}]
\PY{c+c1}{\PYZsh{} checkpoint.restore(tf.train.latest\PYZus{}checkpoint(checkpoint\PYZus{}dir))}
\end{Verbatim}
\end{tcolorbox}

    \hypertarget{crear-el-gif}{%
\subsection{Crear el GIF}\label{crear-el-gif}}

    \begin{tcolorbox}[breakable, size=fbox, boxrule=1pt, pad at break*=1mm,colback=cellbackground, colframe=cellborder]
\prompt{In}{incolor}{19}{\boxspacing}
\begin{Verbatim}[commandchars=\\\{\}]
\PY{k}{def} \PY{n+nf}{display\PYZus{}image}\PY{p}{(}\PY{n}{epoch\PYZus{}no}\PY{p}{)}\PY{p}{:}
    \PY{k}{return} \PY{n}{PIL}\PY{o}{.}\PY{n}{Image}\PY{o}{.}\PY{n}{open}\PY{p}{(}\PY{l+s+s1}{\PYZsq{}}\PY{l+s+s1}{image\PYZus{}at\PYZus{}epoch\PYZus{}}\PY{l+s+si}{\PYZob{}:04d\PYZcb{}}\PY{l+s+s1}{.png}\PY{l+s+s1}{\PYZsq{}}\PY{o}{.}\PY{n}{format}\PY{p}{(}\PY{n}{epoch\PYZus{}no}\PY{p}{)}\PY{p}{)}
\end{Verbatim}
\end{tcolorbox}

    \begin{tcolorbox}[breakable, size=fbox, boxrule=1pt, pad at break*=1mm,colback=cellbackground, colframe=cellborder]
\prompt{In}{incolor}{20}{\boxspacing}
\begin{Verbatim}[commandchars=\\\{\}]
\PY{n}{display\PYZus{}image}\PY{p}{(}\PY{n}{EPOCHS}\PY{p}{)}
\end{Verbatim}
\end{tcolorbox}
 
            
\prompt{Out}{outcolor}{20}{}
    
    \begin{center}
    \adjustimage{max size={0.9\linewidth}{0.9\paperheight}}{output_37_0.png}
    \end{center}
    { \hspace*{\fill} \\}
    

    \begin{tcolorbox}[breakable, size=fbox, boxrule=1pt, pad at break*=1mm,colback=cellbackground, colframe=cellborder]
\prompt{In}{incolor}{21}{\boxspacing}
\begin{Verbatim}[commandchars=\\\{\}]
\PY{n}{anim\PYZus{}file} \PY{o}{=} \PY{l+s+s1}{\PYZsq{}}\PY{l+s+s1}{dcgan.gif}\PY{l+s+s1}{\PYZsq{}}

\PY{k}{with} \PY{n}{imageio}\PY{o}{.}\PY{n}{get\PYZus{}writer}\PY{p}{(}\PY{n}{anim\PYZus{}file}\PY{p}{,} \PY{n}{mode}\PY{o}{=}\PY{l+s+s1}{\PYZsq{}}\PY{l+s+s1}{I}\PY{l+s+s1}{\PYZsq{}}\PY{p}{)} \PY{k}{as} \PY{n}{writer}\PY{p}{:}
    \PY{n}{filenames} \PY{o}{=} \PY{n}{glob}\PY{o}{.}\PY{n}{glob}\PY{p}{(}\PY{l+s+s1}{\PYZsq{}}\PY{l+s+s1}{image*.png}\PY{l+s+s1}{\PYZsq{}}\PY{p}{)}
    \PY{n}{filenames} \PY{o}{=} \PY{n+nb}{sorted}\PY{p}{(}\PY{n}{filenames}\PY{p}{)}
    \PY{n}{last} \PY{o}{=} \PY{o}{\PYZhy{}}\PY{l+m+mi}{1}
    \PY{k}{for} \PY{n}{i}\PY{p}{,}\PY{n}{filename} \PY{o+ow}{in} \PY{n+nb}{enumerate}\PY{p}{(}\PY{n}{filenames}\PY{p}{)}\PY{p}{:}
        \PY{n}{frame} \PY{o}{=} \PY{l+m+mi}{2}\PY{o}{*}\PY{p}{(}\PY{n}{i}\PY{o}{*}\PY{o}{*}\PY{l+m+mf}{0.5}\PY{p}{)}
        \PY{k}{if} \PY{n+nb}{round}\PY{p}{(}\PY{n}{frame}\PY{p}{)} \PY{o}{\PYZgt{}} \PY{n+nb}{round}\PY{p}{(}\PY{n}{last}\PY{p}{)}\PY{p}{:}
            \PY{n}{last} \PY{o}{=} \PY{n}{frame}
        \PY{k}{else}\PY{p}{:}
            \PY{k}{continue}
        \PY{n}{image} \PY{o}{=} \PY{n}{imageio}\PY{o}{.}\PY{n}{imread}\PY{p}{(}\PY{n}{filename}\PY{p}{)}
        \PY{n}{writer}\PY{o}{.}\PY{n}{append\PYZus{}data}\PY{p}{(}\PY{n}{image}\PY{p}{)}
    \PY{n}{image} \PY{o}{=} \PY{n}{imageio}\PY{o}{.}\PY{n}{imread}\PY{p}{(}\PY{n}{filename}\PY{p}{)}
    \PY{n}{writer}\PY{o}{.}\PY{n}{append\PYZus{}data}\PY{p}{(}\PY{n}{image}\PY{p}{)}

\PY{k+kn}{import} \PY{n+nn}{IPython}
\PY{k}{if} \PY{n}{IPython}\PY{o}{.}\PY{n}{version\PYZus{}info} \PY{o}{\PYZgt{}} \PY{p}{(}\PY{l+m+mi}{6}\PY{p}{,}\PY{l+m+mi}{2}\PY{p}{,}\PY{l+m+mi}{0}\PY{p}{,}\PY{l+s+s1}{\PYZsq{}}\PY{l+s+s1}{\PYZsq{}}\PY{p}{)}\PY{p}{:}
    \PY{n}{display}\PY{o}{.}\PY{n}{Image}\PY{p}{(}\PY{n}{filename}\PY{o}{=}\PY{n}{anim\PYZus{}file}\PY{p}{)}
\end{Verbatim}
\end{tcolorbox}


    % Add a bibliography block to the postdoc
    
    
    
\end{document}
